\documentclass[letterpaper,12pt]{article} % This defines the style of your paper
\usepackage[top = 2.5cm, bottom = 2.5cm, left = 2.5cm, right = 2.5cm]{geometry} 

% The following two packages - multirow and booktabs - are needed to create nice looking tables.
%\usepackage{multirow} % Multirow is for tables with multiple rows within one cell.
%usepackage{booktabs} % For even nicer tables.

% As we usually want to include some plots (.pdf files) we need a package for that.
%usepackage{graphicx} 

% The default setting of LaTeX is to indent new paragraphs. This is useful for articles. But not really nice for homework problem sets. The following command sets the indent to 0.
\usepackage{setspace}
\setlength{\parindent}{0in}

% Package to place figures where you want them.
%usepackage{float}

% The fancyhdr package let's us create nice headers.
\usepackage{fancyhdr}
\usepackage{amsmath}


\usepackage[
backend=biber,
style=alphabetic,
citestyle=authoryear
]{biblatex}

\addbibresource{references.bib} %Imports bibliography file

%%%%%%%%%%%%%%%%%%%%%%%%%%%%%%%%%%%%%%%%%%%%%%%%
% 3. Header (and Footer)
%%%%%%%%%%%%%%%%%%%%%%%%%%%%%%%%%%%%%%%%%%%%%%%%

% To make our document nice we want a header and number the pages in the footer.

\pagestyle{fancy} % With this command we can customize the header style.

\fancyhf{} % This makes sure we do not have other information in our header or footer.

\lhead{\footnotesize MATH339: Assignment 1}% \lhead puts text in the top left corner. \footnotesize sets our font to a smaller size.

%\rhead works just like \lhead (you can also use \chead)
\rhead{\footnotesize Johns} %<---- Fill in your lastnames.

% Similar commands work for the footer (\lfoot, \cfoot and \rfoot).
% We want to put our page number in the center.
\cfoot{\footnotesize \thepage} 


\begin{document}

%%%%%%%%%%%%%%%%%%%%%%%%%%%%%%%%%%%%%%%%%%%%%%%%
% Title section of the document
%%%%%%%%%%%%%%%%%%%%%%%%%%%%%%%%%%%%%%%%%%%%%%%%

\thispagestyle{empty} % This command disables the header on the first page. 

\begin{tabular}{p{15.5cm}} % This is a simple tabular environment to align your text nicely 
{\large \bf COMP361: Numerical Methods} \\
Concordia \\ Winter 2021 \\ A. Krzyzak
\hline % \hline produces horizontal lines.
\\
\end{tabular} % Our tabular environment ends here.

\vspace*{0.3cm} % Now we want to add some vertical space in between the line and our title.

\begin{center} % Everything within the center environment is centered.
	{\Large \bf Assignment 1} % <---- Don't forget to put in the right number
	\vspace{2mm}
	
        % YOUR NAMES GO HERE
	{\bf Amiani Johns 26388620}
		
\end{center}  

\vspace{0.4cm}

%%%%%%%%%%%%%%%%%%%%%%%%%%%%%%%%%%%%%%%%%%%%%%%%
%%%%%%%%%%%%%%%%%%%%%%%%%%%%%%%%%%%%%%%%%%%%%%%%

% Up until this point you only have to make minor changes for every week (Number of the homework). Your write up essentially starts here.

\begin{enumerate}

  \item {
    Figures 1 and 2 show the behaviour of the harmonic sum for various values of N. Note that the N values are plotted on a log scale. In both plots we see that the sum continues to grow as N increases.

    Since \sum_{k=1}^{N} \frac{1}{3^k} is a geometric series with constant ratio \(r=\frac{1}{3}\) less than 1, its sum can be calculated using the formula\cite{geo-series}
    \[ \sum_{k=1}{N} r^k = \frac{r(1-r^N)}{1-r} \]
  }

\end{enumerate}

\medskip
\printbibliography

\end{document}